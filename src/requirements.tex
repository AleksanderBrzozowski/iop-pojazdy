\chapter{Wymagania}
Aplikacja powinna służyć jako baza danych nt. pojazdów różnego rodzaju -- tj. pojazdów osobowych, motocykli oraz
pojazdów dostawczych.

Aplikacja powinna umożliwiać dodawanie, usuwanie, edycje oraz przeglądanie pojazdów. Należy
zapewnić również możliwość wyszukiwania pojazdów.
Dane nt. pojazdów oraz użytkowników aplikacji powinny być zapisywane pomiędzy kolejnymi uruchomieniami aplikacji oraz
powinna być możliwość manualnego wczytania oraz zapisania bazy pojazdów do pliku.

W aplikacji wyróżniamy dwóch typów użytkowników: administratora odpowiedzialnego za dodawanie, usuwanie i
modyfikowanie użytkowników oraz zwykłych użytkowników, którzy będą korzystali z aplikacji.

Aplikacja powinna działać pod systemem Windows na komputerach osobistych oraz zawierać graficzny interfejs.
Aby móc korzystać z aplikacji, powinno być wymagane zalogowanie do niej.

\section{Wymagania funkcjonalne}
\begin{itemize}
    \item Przechowywanie danych pojazdu: marka, model, moc, dopuszczalna masa całkowita, liczba siedzeń, paliwo,
    lata produkcji, pojemność silnika, rodzaj pojazdu (pojazd osobowy, motocykl, pojazd dostawczy);
    \item Wyszukiwanie pojazdu po marce oraz modelu;
    \item Wczytanie oraz zapisanie stanu bazy pojazdów do pliku;
    \item Dodawanie, modyfikację oraz usuwanie pojazdów;
    \item Przechowywanie użytkowników: nazwa użytkownika, hasło, rola (administrator lub zwykły użytkownik);
    \item Odczyt i zapis użytkowników systemu do pliku;
    \item Dodawanie, usuwanie, modyfikacja użytkowników za pomocą konta użytkownika z rolą administratora;
    \item Mechanizm uwierzytelnienia w celu korzystania z aplikacji za pomocą nazwy użytkownika i hasła.
\end{itemize}

\section{Wymagania niefunkcjonalne}
\begin{itemize}
    \item Aplikacja okienkowa dedykowana na system Windows (od wersji 7);
    \item Dane o pojazdach oraz użytkownikach systemu przechowywane są w pliku tekstowym (kodowanie UTF-8) o
    rozszerzeniu $.csv$.
    Każdy wiersz pliku opisuje pojedyńczego użytkownika lub pojazd, dane w wierszu rozdzielane są za pomocą ";";
    \item Hasła użytkowników są szyfrowane za pomocą funkcji skrótu SHA-1.
\end{itemize}