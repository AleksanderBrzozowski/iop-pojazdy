\chapter{Wymagania}

Aplikacja powinna służyć jako baza danych nt. pojazdów silnikowych oraz bezsilnikowych.
Do pojazdów silnikowych zaliczamy:
\begin{itemize}
    \item Samochód ciężarowy,
    \item Ciągnik siodłowy.
\end{itemize}
Do pojazdów bezsilnikowych zaliczamy:
\begin{itemize}
    \item Przyczepa,
    \item Naczepa.
\end{itemize}
Pojazdy te można łączyć w pary (zespoły pojazdów) - samochód ciężarowy i przyczepa oraz ciągnik siodłowy i naczepa.

Aplikacja powinna umożliwiać określanie rodzaju przestrzeniu ładunkowej pojazdów, które ją posiadają. Wyróżniamy
następujące rodzaje ładunków: kontener, skrzynia, cysterna. Każda przestrzeń ładunkowa powinna być opisana swoją
ładownością oraz rodzajem (np. cysterna - 1500 \si{\litre}). Dodatkowo powinna być możliwość dodawania ładunków
do pojazdu.

Aktualny stan aplikacji powinien być zapisywany pomiędzy kolejnymi uruchomieniami aplikacji oraz
powinna być możliwość manualnego wczytania oraz zapisania stanu do pliku.

W aplikacji wyróżniamy dwóch typów użytkowników: administratora odpowiedzialnego za dodawanie, usuwanie i
modyfikowanie użytkowników oraz zwykłych użytkowników, którzy będą korzystali z aplikacji.

Aplikacja powinna działać pod systemem Windows na komputerach osobistych oraz zawierać graficzny interfejs.
Aby móc korzystać z aplikacji, powinno być wymagane zalogowanie do niej.

\section{Wymagania funkcjonalne}
\begin{itemize}
    \item Przechowywanie danych pojazdu: rodzaj, nr rejestracyjny, przestrzeń ładunkowa, pojemność przestrzeni
    ładunkowej, rodzaj przestrzeni ładunkowej,
    \item Dodawanie, usuwanie, edytowanie danych pojazdu,
    \item Przeglądanie pojazdów, zespołów pojazdów, ładunków przypisanych do pojazdów,
    \item Łączenie pojazdów w zespoły pojazdów,
    \item Przydzielanie ładunków do pojazdów posiadających przestrzeń ładunkową,
    \item Wczytanie oraz zapisanie stanu bazy pojazdów do pliku,
    \item Przechowywanie użytkowników: nazwa użytkownika, hasło, rola (administrator lub zwykły użytkownik),
    \item Odczyt i zapis użytkowników systemu do pliku,
    \item Dodawanie, usuwanie, modyfikacja użytkowników za pomocą konta użytkownika z rolą administratora,
    \item Mechanizm uwierzytelnienia w celu korzystania z aplikacji za pomocą nazwy użytkownika i hasła.
\end{itemize}

\section{Wymagania niefunkcjonalne}
\begin{itemize}
    \item Aplikacja okienkowa dedykowana na system Windows (od wersji 7),
    \item Dane o pojazdach oraz użytkownikach systemu przechowywane są w pliku tekstowym (kodowanie UTF-8) o
    rozszerzeniu $.csv$.
    Każdy wiersz pliku opisuje pojedyńczego użytkownika lub pojazd, dane w wierszu rozdzielane są za pomocą ";",
    pierwszy wiersz zarezerwowany do opisu kolumn,
    \item Hasła użytkowników są szyfrowane za pomocą funkcji skrótu SHA-1.
\end{itemize}

\section{Przykłady plików przechowujących stan aplikacji}


\begin{table}[h]
    \centering
    \begin{tabular}{ |c|c|c|c|c|c|c| }
        \hline
        id & rodzaj & nr rej. & prz. ład. & pojemn. prz. ład. & rodzaj prz. ład. & id poj. zespołu \\ \hline
        1 & TRUCK & WI 854A & TRUE & 2000 \si{\kilogram} & CONTAINER & NULL \\ \hline
        2 & CAB{\_}OVER & WE 094RT & FALSE & NULL & NULL & 3 \\ \hline
        3 & SEMITRAILER& WE 43AE & TRUE & 2000 \si{\litre} & CISTERN & 2 \\ \hline
        4 & TRAILER & WT 676FG & TRUE & 2000 \si{\kilogram} & CHEST & NULL \\ \hline
    \end{tabular}
    \caption {Pojazdy}
\end{table}

\begin{table}[h]
    \centering
    \begin{tabular}{ |c|c|c|c| }
        \hline
        id & nazwa & ilość & id pojazdu \\ \hline
        1 & benzyna & 1500 \si{\litre} & 3 \\ \hline
        2 & komputery & 200 \si{\kilogram} & 1 \\ \hline
        3 & telewizory & 300 \si{\kilogram} & 1 \\ \hline
    \end{tabular}
    \caption {Ładunki}
\end{table}

\begin{table}[h]
    \centering
    \begin{tabular}{ |c|c|c| }
        \hline
        login & hasło & rola \\ \hline
        abrzozo1 & 19b58543c85b97c5498edfd89c11c3aa8cb5fe51 & USER \\ \hline
        admin & 40bd001563085fc35165329ea1ff5c5ecbdbbeef & ADMIN \\ \hline
    \end{tabular}
    \caption {Użytkownicy}
\end{table}
