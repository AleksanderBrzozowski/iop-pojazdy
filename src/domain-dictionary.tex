\chapter{Słownik dziedziny problemu}
\begin{itemize}
    \item \textbf{Pojazd} -- środek transportu przeznaczony do poruszania się po drodze.
    \item \textbf{Pojazd silnikowy} -- pojazd posiadający silnik, pozwalający mu na przemieszczanie się.
    \item \textbf{Pojazd bezsilinkowy} -- pojazd, który nie posiada silnika, do poruszania się musi być ciągnięty przez
    inny pojazd.
    \item \textbf{Pojazd ciężarowy} -- pojazd przeznaczony konstrukcyjnie do przewozu ładunków i osób, posiadający
    silnik.
    \item \textbf{Ciągnik siodłowy} -- pojazd samochodowy przeznaczony główne do ciągnięcia innych pojazdów drogowych,
    które nie posiadają napędu własnego (przede wszystkim naczep).
    \item \textbf{Przyczepa} -- pojazd drogowy przeznaczony do transportu ładunków, przeznaczony do bycia ciągnionym
    przez pojazd samochodowy.
    \item \textbf{Naczepa} -- rodzaj przyczepy przeznaczony do transportu rzeczy, bez przedniej osi i zaprojektowany
    tak, że część naczepy wraz z ładunkiem spoczywa na tylnej osi ciągniku siodłowym.
    \item\textbf{Skrzynia} -- rodzaj przestrzeni ładunkowej, która składa się z podłogi oraz otwieranych burt oraz
    klapy z tyłu.
    \item \textbf{Cysterna} -- rodzaj przestrzeni ładunkowej służący do przewożenia cieczy.
    \item \textbf{Kontener} -- rodzaj przestrzeni ładunkowej, zazwyczaj metalowa skrzynia, służąca do przewozu drobnicy.
    \item \textbf{Zespół pojazdów} -- połaczenie pojazdu silnikowego oraz bezsilnikowego, które pozwala uzyskać
    większą przestrzeń ładunkową (lub w ogóle uzyskać przestrzeń ładunkową).
    \item \textbf{Administrator} -- użytkownik posiadający uprawnienia administratora.
    \item \textbf{Uwierzytelnienie} -- proces polegający na potwierdzeniu zadeklarowaej tożsamości podmiotu.
\end{itemize}